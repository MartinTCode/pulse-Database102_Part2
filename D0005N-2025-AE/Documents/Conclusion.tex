In this report, we have thoroughly examined various aspects of database management systems (DBMS) 
and their applicability to a Census Bureau's needs. 
We viewed the different compontent of a BI-architecture and their applicability for a Census Bureau.

% BI-architecture elements
We concluded that data marts and sandboxes are an important part of a BI architecture,
as they provide a structured environment for data analysis and experimentation while still being restrictive and carefull in their implementation.
Data mining was concluded to be a crucial part of a BI architecture,
as it enables the discovery of patterns and trends in large datasets.

%analysis of each DBMS
Through a detailed comparison of ClickHouse, SingleStore, and PostgreSQL, 
we have evaluated their performance based on criteria such as scalability, 
storage, cost, security, and latency.
Our analysis revealed that each DBMS has its strengths and weaknesses. 
ClickHouse emerged as the best overall choice due to its superior performance in 
security and latency, which are critical for a Census Bureau. 
Its open-source nature and cost-effectiveness further enhance its appeal. 
SingleStore, while excelling in scalability and storage, 
falls short in terms of cost and security compared to ClickHouse. 
PostgreSQL, despite being a robust and reliable DBMS, lags behind in several key areas, 
making it the least recommended option for this specific use case.



% AI integration
We also explored the challenges and solutions for integrating AI into 
a conventional BI architecture. 
By proposing the addition of a data lake and the use of ELT processes, 
we addressed the need for handling large volumes of unstructured data. 
The integration of vector databases and enhanced security measures further ensures that 
the architecture is optimized for AI-enhanced analytics.

% interview
From the interview discovered the concept of Data Lake House (DLH) and tools used in its implementation,
in the future it would be interesting to explore in greater detail the possibility of implementing a DLH in a Census Bureau.

In conclusion, the best DBMS for a Census Bureau, 
considering the importance of security, latency, and cost, is ClickHouse. 
However, it is essential to adopt a balanced approach, 
leveraging the strengths of each DBMS where applicable, 
to build a robust and efficient data management system. 
The proposed architectural changes will enable the Census Bureau to harness the power of AI, 
ensuring that it remains at the forefront of data-driven decision-making.
Dwelving deeper into the subjects uncovered in the interview before doing so could 
however greatly change the design of the warehouse. 
