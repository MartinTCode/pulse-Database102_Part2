Since a sandbox is a separate environment and store non-permanent data\cite{l2video}, meaning it's data will not have effect on 
the production environment, it can be experiemnted with freely. Sandboxes are also an environment made for sophisticated analysis \cite{l2video}.
These characteristics of sandboxes serves the purpose of facilitating innovation and hypothesis testing.
As such, government analysts and data scientistics at a census burou can experiement with new data models, advanced tools such as machine learning
without the risk of affecting the production environment. The fact that the tools used for analysis of data in the sandbox is freely coupled \cite{l2video} also makes the use of advanced tools simpler.
This is important as it facilitates the development of new models and tools that can be used to improve decision-making processes
without the very large, often complex data structure of a census burou serving as a barrier to innovation.

Having a data warehouse can also serve for cost efficiency purposes, as they can test new BI applications and reports before implementing it into the data warehouse.
Perhaps data engineers or analytics want to see if an external source would complement their data warehouse, they can then test it in the sandbox first.

An analytics sandbox does not rely on batch processing with ETL processes on regular schedules like a data warehouse does \cite{l2video}.
Because of this, with using for example a continous stream of data, 
a sandbox can be used to test real-time or near real-time data processing.
This can be crucial for decision-making a census bureau might need to do in times of crisis, 
such as natural disasters where time is of the essence. 
The usability of advanced tools such as machine learning can also be crucial in these situations, 
which the sandbox would support better than a data warehouse \cite{l2video}.

There are most likely more usability of sandboxes in government decision-making,
but the above mentioned are probably some of the most important ones.

However, the use of sandboxes in government decision-making is not without its cons. It takes substantial skill, storage and computer resources to implement \cite{l2video}.
Because of their free form nature (not defined by business rules), it can be difficult to manage and keep track of the data in the sandbox, possibly leading to data governance issues.
Because of their experimental nature, that is not being highly governed and structured like a data warehouse \cite{l2video}, the data might contain sensitive information, and external sources could be highly faulty or malicious.
This creates a scenario where expertise and care is necessary when using a sandbox, which in turn demands security measures to be rigid and well implemented. Security measure such as these are not free to implement and maintain, thus adding to the cost.