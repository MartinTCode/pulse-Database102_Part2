
\textbf{NEEDS TO BE DONE: CHECK TEXT TO MODIFIED TABLES; CHANGE DOMAIN TO ALIGN W MODIFIED DATA TYPE; CHANGE NAMES IN SOCIOECONOMICS IN STAR SCHEMA}
\section{Logical design of a data warehouse for Census Bureaus}
We have designed a limited dimensional model that could be implemented by a Census Bureau. 
The model is classified as star schema as it has only fact table which is connected to a number 
of denormalized dimension tables (Coursebook, 32.4). The theme chosen for this star schema was 
residence permits in Sweden, which constitutes a part of migration and which among Census Bureaus 
belongs to the subject area “Population”. Depending on the scope and perspective, residence permits 
potentially also connects to “Living conditions” and “Housing”. The motivation of us to design a DW 
about residence permits is mainly to obtain a statistical tool that creates insights about persons 
who have been granted residence permits in Sweden. As to first gain a better understanding of common 
features among the target group, and then to help local, regional and national decision makers in 
Sweden to improve the overall lives of persons with residence permits. 

The star schema that the group created consists of the fact table ResidencePermits, whose primary 
key is a composite key consisting of five foreign keys stemming from each dimension table. The fact 
table contains three fact attributes or facts, which are: permanentResPermits, temporaryResPermits and 
totalResPermits. As the table name and fact attributes suggest, they refer to the quantitative number 
of residence permits granted to individuals in a certain country, in this example in Sweden. Here the 
residence permits have been divided into two additive types, permanent and temporary. Also, total 
permits has been included, which is registered separately but which should equal the sum of the former two types. 

The fact table has been accompanied by five dimension tables. Each one of these contains one surrogate 
type of primary key together with five attributes relating to the dimension in question. The measures 
of the attributes vary within and across dimension tables, but most of them are qualitative. All five 
dimensions Time, Domicile, Background, Socioeconomics and Living –- including their attributes -– have 
relationships with ResidencePermits as the fact being analysed. It is worth noting that the individual 
facts in the fact table are independent from all other attributes in the dimension tables, except for the 
two attributes residencePermitCat and residencePermitDate, which are implicitly interrelated with each of 
the three facts (permanentResPermits, temporaryResPermits and totalResPermits). 
\newpage
\subsection{Fact table and dimension tables (star schema)}
\begin{figure}[h] % "h" means "here"
  \centering
  \includegraphics[width=1.0\textwidth]{Figures/Q4_StarSchema.jpg}
  \label{fig:my_image}
\end{figure}
\newpage
\subsection{Attribute descriptions for logical design}
% ResidencePermits (Fact table)
\begin{table}[htbp]
  \centering
  \caption{ResidencePermits (Fact table)}
  \begin{tabular}{|l|l|l|}
    \hline
    \textbf{Attribute} & \textbf{Domain} & \textbf{Data type} \\
    \hline
    timeID & Foreign Key & INT \\
    \hline
    foreignID & Foreign Key & INT \\
    \hline
    domicileID & Foreign Key & INT \\
    \hline
    backgroundID & Foreign Key & INT \\
    \hline
    socioeconomicsID & Foreign Key & INT \\
    \hline
    livingID & Foreign Key & INT \\
    \hline
    permanentResPermits & Number of granted permanent resident permits & INT \\
    \hline
    temporaryResPermits & Number of granted temporary resident permits & INT \\
    \hline
    totalResPermits & Total number of granted resident permits & INT \\
    \hline
  \end{tabular}
\end{table}

% Time (Dimension table)
\begin{table}[htbp]
  \centering
  \caption{Time (Dimension table)}
  \begin{tabular}{|l|l|l|}
    \hline
    \textbf{Attribute} & \textbf{Domain} & \textbf{Data type} \\
    \hline
    timeID & Primary Key, auto increment & INT \\
    \hline
    year & Year & INT \\
    \hline
    quarter & Quarter of a year & SMALLINT(1-4) \\
    \hline
    month & Month & SMALLINT(1-12) \\
    \hline
    week & Week & SMALLINT(1-53) \\
    \hline
    day & Day & SMALLINT(1-31) \\
    \hline
  \end{tabular}
\end{table}

% Domicile (Dimension table)
\begin{table}[htbp]
  \centering
  \caption{Domicile (Dimension table)}
  \begin{tabular}{|l|l|l|}
    \hline
    \textbf{Attribute} & \textbf{Domain} & \textbf{Data type} \\
    \hline
    domicileID & Primary Key, auto increment & INT \\
    \hline
    region & Region & VARCHAR \\
    \hline
    metropolitanArea & Metropolitan area & VARCHAR \\
    \hline
    municipality & Municipality & VARCHAR \\
    \hline
    districtCode & District code & VARCHAR \\
    \hline
    street & Street & VARCHAR \\
    \hline
  \end{tabular}
\end{table}

% Background (Dimension table)
\begin{table}[htbp]
  \centering
  \caption{Background (Dimension table)}
  \begin{tabular}{|l|l|l|}
    \hline
    \textbf{Attribute} & \textbf{Domain} & \textbf{Data type} \\
    \hline
    backgroundID & Primary Key, auto increment & INT \\
    \hline
    yob & Year of birth & SMALLINT(4) \\
    \hline
    gender & Gender & CHAR(1) \\
    \hline
    nationality & Nationality & VARCHAR \\
    \hline
    residencePermitCat & Residence permit category (Work/EU-EES etc) & ENUM \\
    \hline
    residencePermitDate & Date of residence permit & DATE \\
    \hline
  \end{tabular}
\end{table}

% Socioeconomics (Dimension table)
\begin{table}[htbp]
  \centering
  \caption{Socioeconomics (Dimension table)}
  \begin{tabular}{|l|l|l|}
    \hline
    \textbf{Attribute} & \textbf{Domain} & \textbf{Data type} \\
    \hline
    socioeconomicsID & Primary Key, auto increment & INT \\
    \hline
    civilStatus & Civil status & VARCHAR \\
    \hline
    educationLevelDescription & Education level & VARCHAR \\
    \hline
    educationFieldDescription & Field of education & VARCHAR \\
    \hline
    occupationDescription & Occupation category & VARCHAR \\
    \hline
    income & Yearly gross income in 1000 Swedish Kr & DECIMAL \\
    \hline
  \end{tabular}
\end{table}

% Living (Dimension table)
\begin{table}[htbp]
  \centering
  \caption{Living (Dimension table)}
  \begin{tabular}{|l|l|l|}
    \hline
    \textbf{Attribute} & \textbf{Domain} & \textbf{Data type} \\
    \hline
    livingID & Primary Key, auto increment & INT \\
    \hline
    householdSize & Number of persons in the household & TINYINT \\
    \hline
    childrenNumber & Number of children in the household & TINYINT \\
    \hline
    housingType & Type of housing (building type \& ownership type) & VARCHAR \\
    \hline
    housingService & Housing service (service housing/facility etc) & VARCHAR \\
    \hline
    residentialArea & Residential area in square meters, m2 & DECIMAL \\
    \hline
  \end{tabular}
\end{table}