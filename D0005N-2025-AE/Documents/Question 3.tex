Question:\\
\emph{
    How can the Census Bureau justify its investment in data warehousing and mining
technologies?
}\\\\

\emph{[00:09:56] then third is justify the investment do you think the investment that governments make worldwide
and and and they do actually um in data warehousing and data mining is justified is the return on investment
up in the case i explained some benefits but i also need you not only to take a copy of that but
to discuss it do you think yes it brings value back to them how do you think would that value be
identified and reaped and attained by those governments}\\\\
how do you think would that value be
identified and reaped and attained by those governments
data warehousing and data mining is justified is the return on investment

What to do here?
\begin{enumerate}
    \item motivate why data warehousing is a good idea for the Census Bureau
    \item motivate why data mining is a good idea for the Census Bureau
  \end{enumerate}

\newpage Answers to Question 3: (KIM RESEARCHES REFERENCES)
KIM DOES THIS IN WORD AND THEN PUTS IT HERE. ACCESIBLE VIA LINK:\\
\url{https://ltuse-my.sharepoint.com/:w:/g/personal/sankim-3_student_ltu_se/EVlVl5wzZJpJgSEYamfwzagBWiYV15YG1EIvPHk9yXiOBQ?e=fwckui}

There are several different types of advantages that organisations can gain by making investments in data warehousing (DW) and data mining technologies.
\\\\
Obtaining access to a data warehouse and to associated tools for analysing that data is very valuable for any Census Bureau, 
because it implies a shift from “multiple sources of truths“ to a “single source of truth” \cite{l2video}. 
By implementing DW together with tools, a Census Bureau can consolidate its multiple data sources representing different areas -- 
such as population, society and economy -- into one aggregated pool of data. 
This, in turn, gives a bureau more comprehensive views together with entirely new insights on the data it possesses. 
A bureau can thus discover new patterns in its data but, also falsify earlier believes it had about certain data. 
This is a very important issue, 
as it can be assumed that a typical challenge for Census Bureaus is that they and their customers struggle with multiple information silos, 
many of which are disintegrated rather than integrated.
\\\\
Data and information have become critical success factors for almost all organizations. 
And DW together with its tools supplies its users with entirely new and usually very valuable information \cite[chapter~31.1.3]{CourseLitt}. 
Consequently, one of the foremost arguments for a Census Bureau to opt for DW and data mining is that it can provide a competitive advantage \cite{l2video}. 
Not only for the bureau, but also for the government it serves and for other parties utilizing the information and services provided by the bureau. 
Considering that DW and data mining as technologies have become increasingly popular during the past one-two decades, 
it is perhaps more appropriate to state that these technologies are almost inevitable investments for any Census Bureau today. 
Without DW and analytical tools such as online analytical processing (OLAP) and data mining, 
a Census Bureau would not be able to serve its customers to the full. 
At the same time, it would have a competitive disadvantage towards its national and international peers.
\\\\
As data warehousing and analytical tools are expected to result in new information and a better understanding among the users,
it should eventually also lead to better decisions \cite[chapter~31.1.3]{CourseLitt}. 
This would be a particularly important consequence for an organization like a Census Bureau, 
which serves a broad group of stakeholders covering governments, 
various public and private organizations and also citizens. 
Making better decisions, based on more accurate information, would benefit all parties that utilize the DW and -tools of a Census Bureau. 
The positive effects could on an aggregate level, for the whole society, become substantial.
\\\\
An apparent gain that comes with DW and its tools is the flexibility they provide to those using them. 
Whereas traditional databases such as OLTP-systems are designed to handle fairly simple but high-volume data transactions, DW:s can handle many
different types of queries -- also those being very complex, detailed or multi-conditional \cite[chapter~31.1.4]{CourseLitt}. 
DW would thus be a valuable change for all parties acquiring data from Census Bureaus, 
as it would broaden the scope of queries that are possible for them. 
Moreover, acquiring data though these queries would for Census Bureau users also become easier and faster than before.
\\\\
The technologies of data warehouses have evolved steadily over the years, 
and one area where this applies is the time delays between data in operational sources and in DW:s \cite[chapter~31.1.6]{CourseLitt}. 
The time delays have been reduced and DW:s have evolved towards what can be called real-time (RT) or near real-time (NRT) data warehouses (Coursebook, 31.1.6; Lecture 2-20250213AE). 
This means that data in DW:s is nowadays more current than it was before, 
which in turn shortens the time cycles for making informed decisions. 
Making faster decisions enabled by modern DW technologies would be beneficial for a Census Bureau and its stakeholders, 
as faster decisions can provide decision makers with competitive advantages.
\\\\
One operational gain that can be obtained from DW and data mining is that the implementation process enables and often motivates organizations to
unify terms and definitions for different types of data \cite{l2video}. 
All data that is extracted, transferred and loaded into a DW needs to be harmonized so that it can be used for analyses. 
This would be beneficial for a Census Bureau, which typically obtains and manages data from multiple sources, 
but where uniform data can differ in terms or definitions. 
Unified terms and definitions for same data would improve the quality of the data itself and the Census Bureau operations overall.
Data warehousing could potentially enable a Census Bureau to combine and analyse multiple sets of data that it has not analysed before.
\\\\ 
Altogether it appears clear that investing in data warehousing and in tools such as OLAP or data mining can generate many positive yet 
different effects for a Census Bureau. 
The effects described and elaborated on above are primarily operational, as they relate to things such as more and better information,
and better and faster decision-making. 
If or when this is the case, it should also have financial implications for a Census Bureau and its customers.
On a general level, implementing DW and its tools have in the past been very profitable investments. 
These investments have showed return rates (ROI) of several hundred percent,
together with payback periods measured in months rather than years \cite[chapter~31.1.3]{CourseLitt}, \cite{l2video}. 
Although a Census Bureau itself is a non-profit organization, 
it is justified that it invests in DW technologies as to improve the profitability of its public and private customers. 
Although such investments would expand the budget of a Census Bureau in the short term, 
in the longer term it would most likely be counterweighted by higher tax revenues from society.