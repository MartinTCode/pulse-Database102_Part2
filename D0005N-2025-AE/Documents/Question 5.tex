\textbf{\underline{Motives as to answer yes:}}\\
\textbf{Enhanced performance and faster queries}\\
DM are needed since a Census Bureau is dealing with large and complex data warehouses serving very different subjects.\\
DM:s are specialized thus giving better performance and faster query execution \cite[p.1242]{CourseLitt}.
\\\\
\textbf{Subject-Object analysis:}\\
Immigration might be related to many different subjects.\\
For example, focused employment data is needed for the labour market agency;\\
demographic and disease-related datasets might be needed for the health department.\\
Separating these subject-object relationships into different DM gets rid of overhead\cite[p.1243]{CourseLitt}.
\\
\textbf{cost-effective and reduced complexity:}\\
DM:s can be implemented incrementally, thus reducing project risk and cost \cite[p.1259]{CourseLitt}.
\\\\
\textbf{Support for Distributed Decision-Making:}\\
A well-designed DM structure enables each access stakeholder a curated dataset suited for their specific needs \cite[p.1243]{CourseLitt}.\\
As such, a DM supports distributed decision-making, making the use of the system more efficient.
\\\\
\textbf{Scalability and Flexibility:}\\
DMs segment information into more manageable parts \cite[p.1243]{CourseLitt}. 
This makes handling increasing data-loads more flexible and thus scability easier than in a DW.
\\\\
\textbf{\underline{Motives as to answer no:}}\\
\textbf{Enterprise-Wide Data Access (Single Source of Truth):}\\
DM:s store specialized data for certain subject-object analysis cases \cite[p.1243]{CourseLitt}. 
This may cascade into a creation of many DM:s in such a big and diverse use-case scenario as a Census Bureau since there are many of these cases.
Since a DM has it's own data for this purpose, a DM may, if not properly integrated, create data silos.
Data silos can lead to incosistent data across different agencies creating incongruencies and reliability problems.
Thus they can potentioally violate the \textit{Data Consistency Principle} \cite{l3video}.
A centralized EDW on the other hand ensures that all users access the same data, thus avoiding this problem.
\\\\\textbf{Maintenance and Redundancy Challenges:}\\
DM:s require consistency and synchonization with the main EDW, which significantly increases maintenance cost and complexity \cite[p.1230]{CourseLitt}.
Without DM:s this problem is not present at the same degree.\\


\textbf{\underline{Use data marts, but not indiscriminately or excesivelly:}}\\
- Central EDW should remain the primary data repository to avoid data silos and fragmentation due to duplicate data\\
- High priority departments that require specialized analytics should have DM:s.\\
\\Data marts can provide valuable performance benefits, specialized analytics, and flexibility, 
particularly for a large and diverse organization like a Census Bureau. 
However, their overuse can lead to fragmentation, redundancy, and data silos, 
undermining data consistency and governance. Thus, a balanced approach is required, 
ensuring that data marts are deployed strategically rather than indiscriminately.
