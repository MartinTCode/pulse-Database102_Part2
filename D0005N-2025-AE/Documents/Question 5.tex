Question:\\
\emph{
    Do you think that there exists a need for data marts? Justify either the Y or the N.
}\\\\
\emph{[00:12:27] then five do you think they need a data mart uh once again that's not only i
ask no question but you need to motivate the answer yeah some students might say yes
there are different geographies they need to have autonomy no this is an edw that's sufficient i
i coupled it with a sandbox on top or any variation of those so so this is what i want you to put
as a motivation here to answer the question on data marts there is no here model answer i must say so if one
of you thinks yeah what is the right answer is it marked or without if you say i like suggest an advocate data mart
data mart and give the right explanation and somebody else says i advocate not having a data mart you both
might be right but one thing that i also i also check for is consistency what do i mean by consistency
here we have a layered bi architecture bi for business intelligence in question one and on which you have put
and suggested components let's say for example you suggested the data mart and then you come to
question number five and then you advocated against the data mart this is inconsistency um that might
lead me to send you a comment yeah i cannot accept that because you have just said you are pro data mart and
then uh four questions later you say i'm against data mart for the government agency yeah how do i marry this
to that you need to have consistency um so also in question one there is no model answer if you
say they need edw and mark and sandbox and olab and data mining yeah that might be okay and if somebody
else but as long as you put the right motivation and sequencing etc if somebody else say they need
all only edw and then they don't need that model because of this one two three yeah that also could be
right uh so it's about the motivation to the answer and the consistency across the answers
is what i also check for}\\\\

What to do here?
\begin{enumerate}
    \item possible uses of data marts
    \item pros of using data marts in this case
    \item cons of using data marts in this case
    \item summarize and answer yes or no
    \item motivate the yes/no using the already listed pros/cons
  \end{enumerate}

\newpage Answers to Question 5: (MARTIN RESEARCH REFERENCES)\\\\
\textbf{\underline{Motives as to answer yes:}}\\
\textbf{Enhanced performance and faster queries}\\
DM are needed since a Census Bureau is dealing with large and complext data warehouses serving very different subjects.\\
DM:s are specialized thus giving better performance and faster query execution \cite[p.1242]{CourseLitt}.
\\\\
\textbf{Subject-Object analysis:}\\
Immigration might be related to many different subjects.\\
For example, focused employment data is needed for the labour market agency;\\
demographic and disease-related datasets might be needed for the health department.\\
Separating these subject-object relationships into different DM gets rid of overhead\cite[p.1243]{CourseLitt}.
\\
\textbf{cost-effective and reduced complexity:}\\
DM:s can be implemented incrementally, thus reducing project risk and cost \cite[p.1259]{CourseLitt}.
\\\\
\textbf{Support for Distributed Decision-Making:}\\
A well-designed DM structure enables each access stakeholder a curated dataset suited for their specific needs \cite[p.1243]{CourseLitt}.\\
As such, a DM supports distributed decision-making, making the use of the system more efficient.
\\\\
\textbf{Scalability and Flexibility:}\\
DMs segment information into more manageable parts \cite[p.1243]{CourseLitt}. 
This makes handling increasing data-loads more flexible and thus scability easier than in a DW.
\\\\
\textbf{\underline{Motives as to answer no:}}\\
\textbf{Enterprise-Wide Data Access (Single Source of Truth):}\\
DM:s store specialized data for certain subject-object analysis cases \cite[p.1243]{CourseLitt}. 
This may cascade into a creation of many DM:s in such a big and diverse use-case scenario as a Census Bureau since there are many of these cases.
Since a DM has it's own data for this purpose, a DM may, if not properly integrated, create data silos.
Data silos can lead to incosistent data across different agencies creating incongruencies and reliability problems.
Thus they can potentioally violate the \textit{Data Consistency Principle} \cite{l3video}.
A centralized EDW on the other hand ensures that all users access the same data, thus avoiding this problem.
\\\\\textbf{Maintenance and Redundancy Challenges:}\\
DM:s require consistency and synchonization with the main EDW, which significantly increases maintenance cost and complexity \cite[p.1230]{CourseLitt}.
Without DM:s this problem is not present at the same degree.\\


\textbf{\underline{Use data marts, but not indiscriminately or excesivelly:}}\\
- Central EDW should remain the primary data repository to avoid data silos and fragmentation due to duplicate data\\
- High priority departments that require specialized analytics should have DM:s.\\
\\Data marts can provide valuable performance benefits, specialized analytics, and flexibility, 
particularly for a large and diverse organization like a Census Bureau. 
However, their overuse can lead to fragmentation, redundancy, and data silos, 
undermining data consistency and governance. Thus, a balanced approach is required, 
ensuring that data marts are deployed strategically rather than indiscriminately.

\newpage 
% see how to write a mathematical formula below (note: this line is marked as comment!)
Let $a$ and $b$ be distinct positive integers, and let $c = a - b + 1$

$\mu = a + b $


$\Omega = a - b $

$y = c_2 x^2 + c_1 x + c_0 $

The roots of a quadratic equation are given by:

\begin{equation}
x = \frac{-b \pm \sqrt{b^2 - 4ac}} {2a}
\end{equation}

where $a$, $b$ and $c$ are \ldots
