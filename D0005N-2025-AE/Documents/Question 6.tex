Segregation has for several years or even decades been a societal problem in many EU-countries. 
It can be measured in many ways, but common indicators for it include unemployment, no ongoing education, 
a low level of education and a high(er) demand for health and social services. 
Segregation is a nuisance for the individuals and a cost for the society. 
Segregation exists among almost all groups of people, 
but as a phenomenon it is more common among persons who have moved into their respective countries from abroad. 
Although the general awareness of segregation as a societal challenge has been high for quite a while, 
not so many countries have succeeded in diminishing it by increasing integration among segregated groups. 
Sweden is one of several EU-countries having and working with segregation in society.  

As a group we believe that an EU government, 
such as that in Sweden, would be able to counteract segregation through a more extensive utilization of data mining technologies. 
This should be possible - as we assume that the local census bureau in Sweden has implemented a data warehouse (DW), 
alternatively data mart(s), together with appropriate data mining tools. 
We present the following scenario. 

Segregation: Persons who have been granted residence permits in Sweden typically have foreign backgrounds. 
They are therefore a good approximation of the group of people that has one of the highest risks of experiencing segregation. 
According to the public debate and the academic community, 
certain societal and individual factors affiliated with e.g. domicile (location of home), 
non-native background and living conditions are associated with higher levels of segregation.  

Based on our answer in case question Q4 - where we introduced a DW on granted resident permits - 
we believe that the Census Bureau together with decision makers in Sweden would be interested in 
receiving answers to the following questions: 
\begin{itemize}
  \item Which individual and societal factors are key drivers for segregation among persons that have been granted residence permits? 
  \item Which groups of people having been granted residence permits face the highest risk of becoming segregated? 
  \item How do the profiles and segregation risks differ between persons with temporary contra permanent residence permits? 
  \item Can those groups of people with the highest segregation risk be identified beforehand? 
\end{itemize}

Obtaining answers to or at least more insights into the questions above would help the Census Bureau and authorities Sweden in many ways.
\textbf{First}, it would be valuable for them to gain a better understanding of exactly what factors entail segregation among people 
having been granted residence permits. 
\textbf{Second}, based on the first answer, 
the CB and authorities would be able to segment residence permit holders as to identify those subgroups having 
the highest risk-profiles for segregation. 
\textbf{Third}, they might make new discoveries about how temporary residence permit holders differ from those with permanent permits, 
and perhaps also about why. And \textbf{fourth}, they could possibly even be able to predict which subgroups and individuals will 
become segregated unless some corrective actions are taken. 
All these insights would help the Swedish government and local authorities to impede segregation among residence permit holders - 
a heterogenous yet collectively a highly exposed group of people.  

Our group thinks that the DW on granted resident permits we presented in Q4 contains almost all of the data needed conducting a data mining
scenario on Segregation-Residence permit receivers. 
We recommend that a new attribute (variable) about participation in education is added to one of the dimension tables (Socioeconomics). 
The attribute/variable should be in ordinal scale as to register different educational levels. 
Education would thus be one of two dependent variables that act as proxies for segregation. 
The attribute Income (table Socioeconomics) would constitute the second dependent variable. 
Income as well would be applied on an ordinal scale with different income levels. 
The independent variables for segregation, or the explanatory factors, 
can all be found in the dimension tables of the star schema for the DW. 
Here we do not outline the independent variables specifically, 
we just state that they should connect to the dimension table themes Domicile (Location/District of home), 
Background (Individual characteristics), Socioeconomics (Education \& Occupation) and Living (Household \& Accommodation).
Different amounts and combinations of independent variables should be used against the dependent variables when 
investigating segregation with different data mining techniques.  

As a group we believe \textbf{there are several possible data mining techniques that could fit our problem scenario}. 
The first (1) step of identifying key drivers for segregation could be addressed by applying the data mining task type link analysis and 
the technique associations discovery \cite[chapter~34.2]{CourseLitt}. 
This technique is categorized as unsupervised learning,
as the DW data used would not be labelled in any way \cite{l6video}.  

Once associations between key factors and segregation have been found, 
these factors could be utilized in the second (2) step. 
Here the data mining task type is database segmentation, 
i.e. to segment the data in the DW on granted resident permits \cite[chapter~34.2]{CourseLitt}. 
Although such clustering techniques apply unsupervised learning, 
the segmentation of DW data could be made at least partly based on the segregation factors identified in step 1. 
Such a constellation would imply a semi-supervised learning technique \cite{l6video}. 
The purpose of this step would be to obtain a set of externally diverge but 
internally uniform clusters/segments of people having been granted resident permits. 
The main interest would be in identifying clusters exhibiting the highest segregation risk. 
The obtained clusters could also reveal potential differences between the two subsets of data - 
holders of temporary versus permanent residence permits.  

The third (3) step, which aims to predict which clusters or individuals therein will likely become segregated, 
is needed because the variable values for the individuals in the data are dynamic. 
E.g. their employment status, domicile or civil status can and often do change over time. 
The data mining task in this step is called predictive modelling, and this is a certain type of supervised learning \cite[chapter~34.2]{CourseLitt}. 
Such techniques can be challenging as they require a training phase and a consequent testing phase, 
and the training phase and data need to be sufficiently long and extensive. 
Predictive modelling and its specific technique classification can thus be difficult to conduct on that DW data which 
covers persons being granted only temporary resident permits. The training phases may not be long enough for that data.    