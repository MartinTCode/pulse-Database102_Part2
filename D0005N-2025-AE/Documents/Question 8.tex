Question:\\
\emph{
    AI penetration:\\
- Governments worldwide, especially in Sweden, are interested in adopting AI.
But there is no AI without data.\\ So, in your opinion, how does the layered BI
architecture you proposed in Q.1 need to change to respond to the future AI and
analytics needs of the various government use cases that emerge over time?
}\\

\emph{[00:17:15] question eight of course i will open the floor uh soon after i
finish explaining those questions to you to ask um any question that you might have is about ai penetration
uh so so here we all know that there is a huge interest in ai uh worldwide and specifically some
some governments are putting uh huge investments and plans on this including the swedish government
yeah so in your opinion does the layered bi architecture you provided in question number
one need to change to the future ai and analytics needs or no you took that into account because i also
made that relationship and statement in the beginning i said that yeah there is no ai without data
yeah so here when we set that foundation um to government agencies to the architecture of those government
agencies do you think that this is yeah it took into account the plans of the government to penetrate
more and more use cases using ai yeah or no i really need to rethink and probably add another component
to my layered architecture to cater to the needs of governments when they plan for more use cases
of ai yeah}\\

What to do here?
\begin{enumerate}
    \item Question 1 first
  \end{enumerate}

\newpage Answers to Question 8:
The big challenges we see in implementing AI in a conventional BI architecture as shown in this course is that AI needs massive amount of data, 
and it ought to be unstructured data \cite{IBM_Structured_vs_Unstructured_Data}.
The conventional BI architecture as shown in this report is designed to handle unstructured data at such a scale \cite{Data_Lakes_Article}.
A way we would solve this would by not using an integrated ETL process, but rather an ELT process, where we extract data from external and internal sources,
load the data into a data lake, and then transform it from the data lake \cite{ETL_vs_ELT} into our current architecture (see \ref{sec:question1}).
This would allow us to store all the data in its raw form, and then use it to train models for AI upon the data-lake.
As such, we would need to add a data lake to the architecture, and connect the sandbox to the data lake to test models and 
algorithms before they are put into production.

We would also recommend employing an OLAP tool with good support for AI integration, 
and use vector databases to store data as to optimize it for AI usability \cite{Oracle_Vector_Database}.
As vector databases can be of a columnar storage structure type \cite{Oracle_Vector_Database}, the data in the BI-architecture would still be optimized for 
storing and querying large amounts of data,
with the expanded benefit of being optimized to store data for AI and AI-enhanced analytics needs.

A data lake is as the name suggest a big pool of unstructured data, and as such,
 data privacy and security should be a top priority when implementing this \cite{AWS_Securing_Protecting_Managing_Data}. 

So in summary, our changes to the architectureq would be to add a data lake, extract and load into that data lake, 
and then transform data from there into our current architecture.
We would also connect the sandbox to the data lake, employ an OLAP tool with good support for AI integration and 
use vector databases of the columnar data type to store data and 
increase security measures to mitigate some security risk from storing large and raw data in a datalake.

