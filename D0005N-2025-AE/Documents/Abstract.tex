This study presents a four-layer business intelligence (BI) architecture tailored for Census Bureaus (CBs), comprising source data, extract-transform-load (ETL) processes, a data warehouse (DW), and end-user applications. The primary components of the DW and end-user layers include an enterprise data warehouse (EDW), multiple data marts (DMs), analytic sandboxes for experimentation, and analytical tools such as OLAP and data mining for queries and reports. Sandboxes provide a controlled environment where CBs and government agencies can explore new data models and apply machine learning before full-scale implementation, supporting (near) real-time data processing. However, managing sandbox data effectively remains a challenge.

The adoption of DW and data mining technologies signifies a paradigm shift from multiple sources of truth to a single source of truth, enhancing decision-making and enabling organizations to gain a competitive edge through improved insights and data consistency. A dimensional model was developed for CBs, using a star schema for residence permit holders in Sweden, with a fact table measuring residence permits and five dimensions covering time, domicile, background factors, socioeconomic factors, and living conditions. While data marts enhance performance and flexibility, their excessive use can result in redundancy and data silos, necessitating a strategic deployment approach.

A data mining scenario was proposed to address societal segregation, leveraging an experimental DW containing data on granted residence permits. The study applied link analysis for association discovery, clustering for segmentation, and classification for predictive modeling, focusing on education and income as dependent variables. Furthermore, a comparative evaluation of ClickHouse, SingleStore, and PostgreSQL was conducted across scalability, storage, cost, security, and latency criteria, ultimately recommending ClickHouse as the preferred DW DBMS due to its open-source nature, security features, and efficiency in handling Census Bureau requirements.

To align with future AI advancements, the proposed BI architecture would need to integrate a data lake linked to the sandbox, allowing model testing prior to deployment. Additionally, an AI-compatible OLAP tool and vector databases are recommended to optimize data usability for AI-driven analytics. Finally, insights from an interview highlighted the emergence of hybrid BI architectures, particularly data lakehouses (DLH), as an alternative to conventional BI implementations, warranting further exploration in future research.