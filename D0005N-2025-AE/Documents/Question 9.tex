Question:\\
\begin{em}
Field exercise:

\begin{itemize}
    \item Contact either a Census Office, a Data Warehouse vendor, or a Data Warehouse
    Customer \{could be a government agency or private business\}.
    
    \item The contact could be anywhere geographically, preferably in Sweden/Nordic
    country.
    
    \item The person should have experience with the data warehouse, e.g., data
    warehouse consultant, ETL consultant, DW administrator, DW designer, DW presales
    engineer, Business/Data Analyst, Data Scientist, or a Manager using the
    data warehouse.
    
    \item Ask the person you will contact five questions:
    \begin{itemize}
        \item What is the value of a data warehouse in decision-making?
        \item Which decisions could be supported via the data warehouse?
        \item What is the role of analytics sandboxes in today's business intelligence environment?
        \item How to select an OLAP tool?
    \end{itemize}
    
    \item The answers to the above five questions could be obtained in a phone interview,
    via email, or face-to-face meeting.
    
    \item Report the organization's selection, the person, their answers, and your
    reflections on the answers you have obtained in your report.
\end{itemize}
\end{em}

\emph{[00:18:45] the last question is about the field exercise we learn things in that course and then
it would be great if we can also hear from experts or specialists consultants so so you need to contact
maybe the census office a data warehouse vendor like those names up here or otherwise microsoft oracle ibm etc any
one of those names including as well and then ask that could be a government also agency using
data warehousing data warehousing or a private sector and it's almost everybody is doing so and then ask
preferably in sweden but it's also okay to have that person from outside the nordics ask different
questions yeah but the person should have expertise with data warehousing or etl or data warehouse
administration data warehouse design uh like a sales engineer a consultant business analyst the data
scientist or a data warehouse manager so so this is like propose the titles for the persons that you could interview
interview and then ask them those questions what is the value of a data warehouse in decision making
which decisions could be supported via the data warehouse what is the role of analytic sandbox in
today's business intelligence environment and how do you select an olap tool then you need to report
the interview and the answers of that person you will interview to the questions and \textbf{then put your
insights has that brought new knowledge to you does this come in line with the knowledge that you have
obtained in the course or is it against what you have learned this has triggered some new knowledge to me
this has motivated me to do more research etc so i want you to add your reflection on the answers that
you have obtained yeah and then that all goes into a report}}\\

What to do here?
\begin{enumerate}
    \item 
  \end{enumerate}


\newpage Answers to Question 9:
\section{Person}
Due to confidentiality agreement with the company, the person's name and the company's name cannot be disclosed. The person is thus described as:
Anonymous data engineer (primarily working in the ETL layer). 

\section{Organization}
This section profiles the organization at which the person with whom we have conducted the interview is working. 
Due to confidentiality agreement with the company, the company's name cannot be disclosed. The company is thus described as:
A big housing development company in Sweden.
\section{Interview Questions}
Here is the summarized answers for the interview. See Appendix~\ref{sec:interview} for the full transcript.
\subsection{What is the value of data warehouse to decision making?}

In the interview a representative from a housing development company emphasized the critical role of their data platform 
in decision-making. While they use a data lakehouse rather than a traditional data warehouse, the platform provides valuable 
insights across multiple areas, including finance, health and safety, sales, and marketing.

For example, project leads access continuously updated reports to monitor cost trends, safety compliance, and project performance. 
This enables them to quickly identify cost overruns and optimize spending. Marketing teams use the platform to track campaign 
effectiveness, assessing ROI across different channels to refine their strategies. Similarly, sales teams analyze data to understand 
which projects sell quickly and which require additional effort, helping optimize pricing and marketing tactics.

Overall, the company's centralized data platform enhances operational efficiency, financial oversight, and strategic planning, 
demonstrating the importance of robust data infrastructure in modern business decision-making.

\subsection{Which decisions you think could be supported via the data warehouse?}

In the interview it was explained how the company's data lakehouse plays a key role in decision-making. While primarily relying on 
internal data, they are actively exploring ways to integrate external market trends to enhance their analytical capabilities.

The platform allows project leads to monitor costs in real time, helping them identify overruns and adjust spending accordingly. 
Marketing teams use data insights to assess campaign performance and optimize ad placements on platforms like Google and YouTube. 
Sales teams analyze trends to determine where additional marketing efforts are needed and whether pricing strategies should be adjusted. 
Additionally, by reviewing past project success, the company refines its future investment and development strategies.

Looking ahead, they aim to incorporate broader housing market data and industry trends, which would provide a more comprehensive basis 
for strategic decision-making.

\subsection{What is the role of analytics sandboxes in today's business intelligence environment?} 

The representative discussed how they do not use a traditional analytics sandbox, but instead operate with flexible environments that 
serve similar functions. Their system is built on a data lakehouse, and they maintain development, test, and production environments 
to support rapid development and experimentation. These environments allow teams to test new hypotheses and analysis methods without 
affecting the production environment.

The company also uses Apache Spark and Parquet files, which enable them to conduct ad hoc analysis in a flexible, low-risk way. The 
use of Spark Notebooks provides the ability to test and visualize data quickly, making it possible to try out different approaches and 
gather insights without disrupting ongoing operations. This setup supports rapid development cycles and allows for fearless experimentation 
in a way that is similar to the benefits provided by an analytics sandbox.

\subsection{How to select an OLAP tool?}

The representative explained that their company does not use an OLAP tool within their data team, but a separate team handles it. When 
selecting an OLAP tool, performance is a key consideration, particularly the ability to store all data in memory rather than on disk, which 
provides significant performance benefits. However, this approach can lead to high costs due to the memory usage.

The representative mentioned that if they were selecting an OLAP tool, they would focus on the development operations and ensure the tool 
supports agile development and works with version control and continuous integration/continuous delivery (CI/CD) processes, which have been 
a challenge for the current tool.

The company is currently exploring alternatives to their existing OLAP tool, considering tools like Spark and Power BI for certain use cases. 
The representative noted that for their team, Spark and notebooks already fulfill many of the functionalities required for ad hoc analysis, 
offering a more flexible and cost-effective solution compared to a traditional OLAP tool.







