\section{The Web Exercise}
This question requires a web exercise,therefore we have decided to add sections, one for each product. 

\subsection{Choice of criteria:}
The criteria we have chosen to evaluate the vendors and why are:
\begin{itemize}
    \item Scalability: This is an important criteria since:
    \begin{quotation}
        "Can we scale up, so this is important [...] the goverment is opening up it's database to the public for example.[...] so scalable means it can support more users and larger data volume"\cite{l1video}".
    \end{quotation}
    \item Storage: Storage is an important criteria since
    the DBMS must efficiently handle large volumes of data, 
    ensuring optimal load performance and storage management to 
    accommodate growing datasets without architectural limitations \cite[p. 1239]{CourseLitt}.
    \item Cost: This is an important criteria since developing a DBMS is a costly endeavour, and the cost of the chosen DBMS can add or substract from this. \begin{quotation}
        "the cost can vary enormously from tens of thousands to millions of dollars due to
    the variety of technical solutions available." \cite[p. 1226]{CourseLitt}
    \end{quotation}
    \item Security: \begin{quote}
        "Security is critical to the data warehouse. To provide the strongest possible
security and to minimize administrative overhead, all security policies are enforced
within the data warehouse."\cite[p. 1309]{CourseLitt}
    \end{quote}
    \item Latency: This is an important criteria since \begin{quotation}
         "the DBMS must be able to handle large, complex queries for key business operations that must complete in reasonable time periods." \cite[p. 1239]{CourseLitt}
    \end{quotation} 
\end{itemize}

\subsection{Vendor 1 - ClickHouse:}
\begin{itemize}
    \item Scalability: ClickHouse allows for multiple nodes which eases horizontal scalability, 
    ClickHouse does not automatically shard, and re-sharding your dataset will require significant compute resources. \cite{clickhouseScaling}
    Given ClickHouse's provided example of a big data warehouse hardware specifications, vertical scaling seems to be well supported \cite{clickhouseScaling}.
    \item Storage: Clickhouse uses a column-oriented storage model, which is efficient for analytical queries \cite{clickhouseStorage}. ClickHouse is not inherently designed for in-memory usage \cite{clickhouseStorage}.
    In the benchmark ClickHouse scores the highest in terms of storage size reduction (See Storage Size under Appendix~\ref{sec:benchmarks}).
    \item Cost: ClickHouse is free to deploy and use, and is open-source \cite{clickhouseCost}.
    \item Security: 
    \begin{itemize}
        \item Data Encryption: Supports data encryption with AES 256 keys and Transparent Data Encryption (TDE) \cite{clickhouseSecurity}.
        \item Access Control: Supports Role-based access control and multi-factor authentication (MFA) \cite{clickhouseSecurity2}
        \item Security Certificates: ISO 27001 compliance, SOC 2 Type II compliance, GDPR and CCPA compliance, HIPAA compliance \cite{clickhouseSecurity2}
    \end{itemize}
    \item Latency: Clickhouse scores the highest in terms of latency (See Hot Run and Cold Run under Appendix~\ref{sec:benchmarks}).
\end{itemize}    
\subsection{Vendor 2 - SingleStore:}
\begin{itemize}
    \item Scalability: allows both vertical and horizontal scaling. Horizontal scaling through nodes \cite{singlestoreScaling}. SingleStore does automatically shard \cite{SinglestoreSharding}.
    \item Storage: Can be modified to have either column- or row-oriented storage \cite{singlestoreStorage}. SingleStore is designed for in-memory usage \cite{singlestoreStorage}.
    In the benchmark SingleStore scores worse than ClickHouse but far better than PostgreSQL in terms of storage size reduction (See Storage Size under Appendix~\ref{sec:benchmarks}).
    \item Cost: SingleStore is charged at individual rates (no publically available fixed fee) \cite{singleStoreCost}.
    \item Security: 
    \begin{itemize}
        \item Data Encryption: SingleStore automatically enforces AES 256-bit encryption \cite{singlestoreSecurity}.
        \item Access Control: SingleStore supports Row-Level Security (RLS) \cite{singlestoreSecurity2}.
        \item Security Certificates: SOC 2 Type 2, HIPAA, CCPA, GDPR \cite{singlestoreSecurity3}.
    \end{itemize}
    \item Latency: SingleStore scores worse than ClickHouse but far better than PostgreSQL in terms of latency (See Hot Run and Cold Run under Appendix~\ref{sec:benchmarks}).
\end{itemize} 
\subsection{Vendor 3 - PostgreSQL:}
\begin{itemize}
    \item Scalability: PostgreSQL can scale horizontally but it is difficult and convoluted to implement \cite{postgresqlHorScaling}.
    PostgreSQL is however known to be able to scale vertically rather well \cite{postgresqlVerScaling}.
    \item Storage: PostgreSQL is traditionally a row-oriented database management system (DBMS), but it supports columnar storage through extensions \cite{postgresqlStorage}.
    In the benchmark PostgreSQL scores the worst by a large margin in terms of storage size reduction (See Storage Size under Appendix~\ref{sec:benchmarks}).
    \item Cost: PostgreSQL is free to deploy and use, and is open-source \cite{postgresqlCost}.
    \item Security:
    \begin{itemize}
        \item Data Encryption: PostgreSQL supports Transparent Data Encryption (TDE) \cite{postgresqlSecurity} and, with the help of an extension, Column-Level Encryption with AES-256 \cite{postgresqlSecurity2}.
        \item Access Control: PostgreSQL supports Role-Based Access Control (RBAC) \cite{postgresqlSecurity3} and Row-Level Security (RLS) \cite{postgresqlSecurity4}.
        \item Security Certificates: None found.
    \end{itemize}
    \item Latency: PostgreSQL scores by far the worst in terms of latency (See Hot Run and Cold Run under Appendix~\ref{sec:benchmarks}).
\end{itemize} 
\section{The Comparison}
In this section we provide a comparison between vendors, therefore we have decided to add sections, one for each criterion. 

\subsection{Criteria 1 - Scalability:}
All three DBMS:s have good support for vertical scaling.
Both ClickHouse and SingleStore are designed to scale horizontally through sharding. ClickHouse does not automatically shard, and re-sharding your dataset will require significant compute resources \cite{clickhouseScaling}. SingleStore does automatically shard \cite{SinglestoreSharding}. PostgreSQL can scale horizontally but it is difficult and convoluted to implement \cite{postgresqlHorScaling}.
\\\\
As such, the three DBMS are ranked as follows, with the best first:

\begin{enumerate}
    \item SingleStore
    \item ClickHouse
    \item PostgreSQL
\end{enumerate}
\subsection{Criteria 2 - Storage:}
Both ClickHouse and SingleStore can use column-oriented storage, which is effecient for OLAP systems as formerly mentioned. PostgreSQL does not natively support columnar storage.
ClickHouse is not inherently designed for in-memory usage \cite{clickhouseStorage}, while SingleStore is designed for in-memory usage \cite{singlestoreStorage}.
As per optimizing storage size, ClickHouse scores the highest, SingleStore scores second, and PostgreSQL scores the lowest by a  far margin.
\\
Here which rankest the highest is not very clear, given that ClickHouse is not designed for in-memory usage, which is a disadvantage in terms of efficiency. However, ClickHouse scores the highest in terms of storage size reduction.
Since the writers of this report are not experts in the field and given the aforementioned uncertainty, we will rank ClickHouse and SingleStore the highest for the sake of fairness. 
\\\\
As such, the three DBMS are ranked as follows, with the best first:

\begin{enumerate}
    \item SingleStore \& ClickHouse
    \item PostgreSQL
\end{enumerate}
\subsection{Criteria 3 - Cost:}
Both PostgreSQL and ClickHouse are free to deploy and use, and are open-source \cite{clickhouseCost} \cite{postgresqlCost}. SingleStore is charged at individual rates \cite{singleStoreCost}.
\\\\
As such, the three DBMS are ranked as follows, with the best first:

\begin{enumerate}
    \item PostgreSQL \& ClickHouse
    \item SingleStore
\end{enumerate}
\subsection{Criteria 4 - Security:}
As per data encryption, all three DBMS:s support AES 256-bit encryption.\\
As per access control, Both SingleStore and PostgreSQL allow for Row-Level Security (RLS), while ClickHouse does not.
ClickHouse is the only DBMS found to support multi-factor authentication.\\
As per security certifications, postgreSQL does not seem to have any, whereas both SingleStore and ClickHouse have several.
However, ClickHouse is the only one who has a \textit{ISO 27001} certification. The \textit{ISO 27001} \textit{"certification demonstrates that an organisation has defined and put in place best-practice information security processes. "}\cite{ISO27001}.
\\\\
As such, the three DBMS are ranked as follows, with the best first:

\begin{enumerate}
    \item ClickHouse
    \item SingleStore
    \item PostgreSQL
\end{enumerate}

\subsection{Criteria 5 - Latency:}
As seen in the benchmark Appendix~\ref{sec:benchmarks}, ClickHouse scores the best in terms of latency, SingleStore scores second, and PostgreSQL scores the worst by a large margin.
\\\\
As such, the three DBMS are ranked as follows, with the best first:

\begin{enumerate}
    \item ClickHouse
    \item SingleStore
    \item PostgreSQL
\end{enumerate}

\section{Conclusion}
In favor of offering a quick overview of the ranking of the different criterias for each DBMS, a table is provided below with scoring per criteria for each DBMS. 
If the DBMS came first in a criteria, it was awarded 3 points, if it came second it was awarded 2 points, and if it came last it was awarded 1 point.

\begin{table}[h]
    \centering
    \begin{tabular}{|c|c|c|c|c|c|c|}
    \hline
    \textbf{DBMS} & \textbf{Scalability} & \textbf{Storage} & \textbf{Cost} & \textbf{Security} & \textbf{Latency} & \textbf{Total} \\ \hline
    ClickHouse & 2 & 3 & 3 & 3 & 3 & 14 \\ \hline
    SingleStore & 3 & 3 & 2 & 2 & 2 & 12 \\ \hline
    PostgreSQL & 1 & 2 & 3 & 1 & 1 & 8 \\ \hline
    \end{tabular}
    \caption{Scoring per criteria for each DBMS}
    \label{tab:my-table}
\end{table}

As per this table and the former chapter we can conclude that PostgreSQL is the least recommended DBMS. However, as per the best DBMS, it is not necessarily as clear of a choice as the overview table (Table~\ref{tab:my-table}) might suggest.
For example, both scalability and security are of utmost importance for a Census Bureau, and ClickHouse is the only DBMS that supports multi-factor authentication and has an \textit{ISO 27001} certification and is thus the better choice when it comes to security. 
Clickhouse is also free to use, and is open-source, which is a big advantage in terms of cost.
However, SingleStore is the better choice when it comes to scalability, and storage, and is the second best choice when it comes to security and latency.

As such, the best DBMS for the Census Bureau is ClickHouse since security and latency are of outmost importance for a census bureau, it being free and open-source is also a big advantage. The open-source aspect has several positive aspects not formerly discussed, such as the ability to modify the source code to better fit the needs of the Census Bureau.
The fact that SingleStore is better at scalability is note-worthy and important to reflect upon, especially given the huge performance demand on a big DBMS such as one employed by a census bureau, but ClickHouse does have rigid scalability support as is and the difference is thus not considered big enough to discredit the advantages that ClickHouse offers.