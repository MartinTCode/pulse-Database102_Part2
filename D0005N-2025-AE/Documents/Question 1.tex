Question\cite{CourseLitt}:\\
\emph{Suggest a layered BI architecture for Census Bureaus to use. Explain components and
justify the use of each. This should be your own, not taken as-is from any other source.}

\indent{ \emph{
 - Hint: a figure is required here, together with motivation to its components and
explanation of the connections!
}}\\\\
\emph{[00:08:15]suggest a layered architecture remember when we studied the data warehouse i presented
different generic architecture options and then your role is to suggest one you cannot copy one of those that i have explained in the course it must be your own
and then that needs a drawing and explanation as well so this is my architecture and this is why for example
i think that they need edw but not a data mark somebody else might come and say
yeah this is my architecture on which there are a number of data marks but there is no edw and third might come
and say yeah i have both edw and data mark and fourth student might come and say and also i think that they need sandbox and etc and then you blend those
uh different um technology options and architecture options that um we have explained so far in the course and you need to justify each of the components
yeah but once again it cannot be one of those that we have explained in the course and also it needs to be your own work}\\\\
What to do here?
\begin{enumerate}
    \item research layered BI architecture
    \item create layered BI architecture diagram
    \item explain each component of the diagram
    \item motivate the design
  \end{enumerate}

\newpage Answers to Question 1:\footnote{Part II Lecture 2-20250213AE, 00:50:00}

\newpage 
\begin{enumerate}
  \item This is the first entry in my list
  \item This is the second entry in my list
  \item This is the third entry in my list
\end{enumerate}

\begin{center}
    \begin{tabular}{ c c c }
        cell1 & cell2 & cell3 \\ 
        cell4 & cell5 & cell6 \\  
        cell7 & cell8 & cell9    
    \end{tabular}
\end{center}

\begin{table}[h]
\centering
    \begin{tabular}{ |c| c| c| }
     \hline
        cell1 & cell2 & cell3 \\ 
        cell4 & cell5 & cell6 \\  
        cell7 & cell8 & cell9 \\  
     \hline
    \end{tabular}\\
    \caption{Insert Table Caption here!}
    \label{table:1}
\end{table}